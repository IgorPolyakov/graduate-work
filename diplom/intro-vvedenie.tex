\newpage
\section{Введение}
%\addcontentsline{toc}{section}{Введение}
%Данное программное обеспечение разрабатывается для задач оценки деформации твёрдого тела.

%Решение задач оценки деформации охватывают широкий спектр применения от биологической до аэрокосмической и в масштабах от микроскопии до крупных промышленных структур.

%В задачах автоматизации данное программное обеспечение может использоваться для автоматизации измерений при проведении экспериментальных исследований по оценке и прогнозированию поведения объектов и конструкций при воздействии нагрузок в полевых условиях и на испытательных стендах.

Оптический метод оценки деформации, основанный на корреляции цифровых изображений, включает два основных этапа: 1) построение поля векторов перемещений и 2) последующий расчет компонент деформации~\cite{pan_intro_one}. Большинство исследований в области разработки алгоритмов построения векторов перемещений направлены на повышение точности и увеличение помехоустойчивости определения смещений~\cite{pan_intro_two, pan_intro_three}, либо увеличение быстродействия.

В современных системах основанных на корреляции цифровых изображений существует проблема выбора размера ядра корреляции. 
%Под ядром корреляции подразумевается площадка изображения, для которой определяется перемещение. 
Заметим, что в методе определения оптического потока площадка корреляции, как правило, соответствует небольшим по площади фрагментам изображения, что обусловлено, прежде всего, необходимостью снижения вычислительных затрат при построении полного поля перемещений, а также обеспечения достаточно высокой плотности векторов поля перемещений~\cite{tom_lyk, Lucas1981}.

Целью настоящей работы является разработка программного обеспечения (ПО), рассмотрение алгоритмов вычисления оптического потока для оценки деформаций поверхностей твёрдых тел, т.е. для получения качественной и количественной оценки процессов, развивающихся в деформируемом твёрдом теле, а также проведение исследований алгоритмов и методов на реальных оптических изображениях.

Область применения ПО:
\begin{itemize}
\item экспериментальное исследование механизмов деформации и разрушения структурно-неоднородных материалов с целью последующего компьютерного моделирования их структуры и свойств;
\item аттестация режимов формирования защитных и упрочняющих покрытий с целью корректировки режимов их нанесения;
\item тарировка приборов неразрушающего контроля и дефектоскопии.
\end{itemize}

%Внедрение работы. Созданное программное обеспечение включено в состав оптико-телевизионного комплекса ``TOMSC'' и используется для проведения исследований различных материалов и сплавов в ИФПМ СО РАН.