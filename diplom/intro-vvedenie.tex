\newpage
\section{Введение}
%\addcontentsline{toc}{section}{Введение}
%Данное программное обеспечение разрабатывается для задач оценки деформации твёрдого тела.

%Решение задач оценки деформации охватывают широкий спектр применения от биологической до аэрокосмической и в масштабах от микроскопии до крупных промышленных структур.

%В задачах автоматизации данное программное обеспечение может использоваться для автоматизации измерений при проведении экспериментальных исследований по оценке и прогнозированию поведения объектов и конструкций при воздействии нагрузок в полевых условиях и на испытательных стендах.

Оптический метод оценки деформации, основанный на корреляции цифровых изображений (именуемый в зарубежной литературе {DIC}~--- digital image correlation), включает два основных этапа: 1) построение поля векторов перемещений и 2) последующий расчет компонент деформации~\cite{pan_intro_one}. Большинство исследований в области разработки алгоритмов построения векторов перемещений направлены на повышение точности и увеличение помехоустойчивости определения смещений~\cite{pan_intro_two, pan_intro_three}, либо увеличение быстродействия.

В современных DIC-системах перед нагружением на поверхности исследуемого материала с помощью двух баллонов краски формируется спекл-картина~\cite{pan_intro_one}. Это позволяет повысить контрастность и обеспечить достоверное определение перемещений.
При этом форма и размер элементов спекла могут существенно влиять на точность и помехоустойчивость измерения смещений.
Погрешность измерения определялась как сумма систематической ошибки, вызванной субпиксельной ошибкой при определении смещений, и случайной погрешности, обусловленной наличием шумов и их уровнем. Показано, что радиус пятен в спекле порядка 3~$\sim$~4 пикселов обеспечивает минимальную ошибку определения смещения.

Помимо выявления оптимального размера элементов спекла, существует проблема выбора размера ядра корреляции. Под ядром корреляции подразумевается площадка изображения, для которой определяется перемещение, и её центр соответствует координатам искомого вектора перемещений. В зарубежной литературе для обозначения размеров фрагментов изображений, участвующих в работе корреляционного алгоритма, применятся термин subset size. Далее в статье будем использовать термин ``размер площадки корреляции'', который по нашему мнению наиболее соответствует термину subset size. Заметим, что в методе определения оптического потока площадка корреляции, как правило, соответствует небольшим по площади фрагментам изображения, что обусловлено, прежде всего, необходимостью снижения вычислительных затрат при построении полного поля перемещений, а также обеспечения достаточно высокой плотности векторов поля перемещений~\cite{tom_lyk, Lucas1981}.

В работе~\cite{pan_intro_four} исследовали проблему выбора размера площадки корреляции и в качестве варианта ее решения предложен подход, в основе которого лежит определение суммы квадратов градиентов интенсивности участков изображения {SSSIG} (Sum of Square of Subset Intensity Gradients). Оценено влияние размера площадки корреляции, а также шума на изображении на ошибку определения перемещений. В~\cite{pan_intro_one} приводятся рекомендации для выбора размера площадки корреляции в зависимости от физического размера образца материала, изображённого на фотографии, а также от разрешения изображения (оптической системы) и размера элементов спекла (пятен). Приведённая информация является скорее рекомендацией по проведению съёмки материалов для обеспечения требуемой точности, чем конкретным алгоритмом выбора размера площадки корреляции.

Целью настоящей работы является разработка программного обеспечения (ПО), рассмотрение алгоритмов вычисления оптического потока для оценки деформаций поверхностей твёрдых тел, т.е. для получения качественной и количественной оценки процессов, развивающихся в деформируемом твёрдом теле, а также проведение исследований алгоритмов и методов на реальных оптических изображениях.

Область применения ПО:
\begin{itemize}
\item экспериментальное исследование механизмов деформации и разрушения структурно-неоднородных материалов с целью последующего компьютерного моделирования их структуры и свойств;
\item аттестация режимов формирования защитных и упрочняющих покрытий с целью корректировки режимов их нанесения;
\item тарировка приборов неразрушающего контроля и дефектоскопии.
\end{itemize}

%Внедрение работы. Созданное программное обеспечение включено в состав оптико-телевизионного комплекса ``TOMSC'' и используется для проведения исследований различных материалов и сплавов в ИФПМ СО РАН.