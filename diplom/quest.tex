\newpage
\ESKDthisStyle{empty}

\begin{center}
 Министерство образования и науки Российской Федерации\\
 Федеральное государственное бюджетное образовательное учреждение высшего профессионального образования\\
 ТОМСКИЙ ГОСУДАРСТВЕННЫЙ УНИВЕРСИТЕТ СИСТЕМ УПРАВЛЕНИЯ И РАДИОЭЛЕКТРОНИКИ (ТУСУР)\\
 Кафедра электронных средств автоматизации и управления (ЭСАУ)\\
\end{center}

\begin{flushright}
 \begin{minipage}{0.4\textwidth}
  УТВЕРЖДАЮ \\
  зав. каф. ЭСАУ \\
  \underline{\hspace{2.5cm}}О.И. Черепанов \\
  "\underline{\hspace{1cm}}"\underline{\hspace{3cm}} 2015г.
 \end{minipage}
\end{flushright}

\vspace{1cm}

\paragraph*{\hfill ЗАДАНИЕ \hfill}

на дипломную работу студенту Полякову Игорю Юрьевичу группы 530, факультета вычислительных систем.

1. Тема работы: Разработка и тестирование программного обеспечения для оценки деформации поверхности твёрдых тел по серии оптических изображений.

2. Срок сдачи студентом законченной работы: "\underline{\hspace{1cm}}"\underline{\hspace{3cm}} 2015г.

3. Исходные данные к работе:

\begin{itemize}
 \item Программное обеспечение для оценки деформации ``Deformation Analysis'';
 \item Дифференциальный алгоритм Лукаса−Канаде.
\end{itemize}
\newpage
4. Содержание расчётно-пояснительной записки (перечень подлежащих разработке вопросов):

\begin{itemize}
 \item введение;
 \item описание алгоритмов;
 \item проектирование программного обеспечения;
 \item тестирование;
 \item заключение.
\end{itemize}

5. Перечень графического материала:

\begin{itemize}
 \item презентация.
\end{itemize}

6. Консультанты по работе

\begin{itemize}
  \item консультант по экономике: доцент кафедры экономики, кандидат \\ экономических наук \\
  \begin{singlespace}
 О.П. Полякова\hfill \underline{\hspace{6cm}} \\
 \begin{flushright} "\underline{\hspace{1cm}}"\underline{\hspace{3cm}} 2015г. \end{flushright}
 \end{singlespace}
 \item консультант по вопросам охраны труда: доцент кафедры РЭТЭМ, кандидат химических наук\\
 \begin{singlespace}
 И.А. Екимова \hfill \underline{\hspace{6cm}} \\
 \begin{flushright} "\underline{\hspace{1cm}}"\underline{\hspace{3cm}} 2015г. \end{flushright}
 \end{singlespace}
\end{itemize}

Задание выдано:

Руководитель: младший научный сотрудник ИФПМ СО РАН, кандидат технических наук \\
\begin{singlespace}
П.С. Любутин \hfill \underline{\hspace{6cm}} \\
\begin{flushright} "\underline{\hspace{1cm}}"\underline{\hspace{3cm}} 2015г. \end{flushright}
\end{singlespace}

Задание принято к исполнению:

\begin{singlespace}
студент И.Ю. Поляков \hfill \underline{\hspace{6cm}} \\
\begin{flushright} "\underline{\hspace{1cm}}"\underline{\hspace{3cm}} 2015г. \end{flushright}
\end{singlespace}