\setcounter{figure}{0}

В результате выполнения работы было разработано программное обеспечение для оценки деформаций поверхности твёрдого тела.  ПО позволяет оценивать оптический поток по серии входных снимков и проводить исследования механизмов пластической деформации на мезоуровне. %Основная отличительная особенность ПО от разработок существовавших ранее в том, что оно позволяет производить расчеты от качественных показателей (ПВС) до количественных результатов в виде компонент деформации с возможностью корректировки промежуточных результатов. Таким образом, ПО объединяет в себе средства для решения нескольких задач. 

При разработке ПО были рассмотрены и реализованы методы вычисления оптического потока на основе алгоритма Лукаса-Канаде, его пирамидальная версия, и итеративная модификация с субпиксельной точностью.

Тестирование ПО показало, что классический алгоритм Лукаса-Канаде локален и не способен определить смещение большее чем область поиска. Пирамидальная версия алгоритма решает эту проблему. %при уровне использующейся в настоящее время вычислительной техники наиболее эффективен корреляционный алгоритм с заданием области расчета коэффициента корреляции, т.к. это наиболее точный метод (из реализованных в ПО) при приемлемых вычислительных затратах. Алгоритмы, оптимизированные по времени расчета, могут использоваться для быстрой предварительной оценки ПВС.
%Предложенный алгоритм нахождения ПВС с субпиксельной точностью показал свою эффективность при расчете компонент тензора дисторсии. Повышение точности определения векторов смещений до долей пиксела позволяет более точно и корректно рассчитать распределения компонент деформации.