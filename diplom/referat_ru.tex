\newpage
\ESKDthisStyle{empty}
\paragraph{\hfill Реферат \hfill}
Выпускная квалификационная работа \ESKDtotal{page} с., \ESKDtotal{figure} рис., \ESKDtotal{table} табл., \ESKDtotal{bibitem} источников, \ESKDtotal{appendix} прил.


ПОЛЕ ВЕКТОРОВ ПЕРЕМЕЩЕНИЙ, КОМПОНЕНТЫ ДЕФОРМАЦИИ, СУБПИКСЕЛЬНАЯ ТОЧНОСТЬ

Целью настоящей работы является разработка программного обеспечения (ПО) для оценки деформаций поверхностей твёрдых тел, а также проведение исследований алгоритмов и методов как на модельных, так и на реальных оптических изображениях.

В работе исследовано влияние метода интерполяции изображений с субпиксельной точностью с использованием итеративного подхода к построению поля векторов перемещений (оптического потока).

Проект выполнен с использованием следующих средств разработки: языка программирования C++(Qt), среды разработки QtCreator 3, TeXstudio 2. Система контроля версий git.

Программное обеспечение, разработанное в ходе данной работы, представлено в приложении.

Отчёт выполнен согласно ОС ТУСУР 01-2013 при помощи текстового процессора \LaTeX.