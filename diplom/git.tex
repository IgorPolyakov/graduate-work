Для разработки программного комплекса решено использовать Git.

Git  — распределённая система управления версиями файлов. Проект был создан Линусом Торвальдсом для управления разработкой ядра Linux, как противоположность системе управления версиями Subversion (также известная как «SVN») \cite{progit}.

Необходимость использования системы версий, очевидна. Так как в группе несколько программистов и тестер, мы имеем:
\begin{itemize}
\item возможность удалённой работы с исходными кодами;
\item возможность создавать свои ветки, не мешая при этом другим разработчикам;
\item доступ к последним изменениям в коде, т.к. все исходники хранятся на сервере github.com ;
\item исходные коды открыты, доступ к ним можно получить доступ в интернет;
\item возможность откатиться к любой стабильной стадии проекта.
\end{itemize}

Основные постулаты работы с кодом в системе Git:

\begin{itemize}
\item каждая задача решается в своей ветке;
\item коммитим сразу, как что-то получили осмысленное;
\item в master мержится не разработчиком, а вторым человеком, который производит вычитку и тестирование изменения;
\item все коммиты должны быть осмысленно подписаны/прокомментированы.
\end{itemize}

Для работы над проектом был использован репозиторий на сервере github.com. Слепок последних изменений из репозитория можно взять по адресу:

git clone git@github.com:IgorPolyakov/graduate-work.git